




% Preamble.tex

%%%
%%% 导言区
%%%

% 防止重复引用
\ifdefined \preambleInc
 % 如果重复引用
\setcounter{lstnumber}{1}
\endinput
\else
\def\preambleInc{}
\fi

%%%%%%%%%%%%%%%%%%%%%%%%%%%%%%%%%%%%%%%%%%%%%%%%

% Document Class 设置
\makeatletter
\ifx \@UseCTEX \undefined
\documentclass[UTF8]{\@DocType}
\usepackage{xeCJK}
\setCJKmainfont{WenQuanYi Zen Hei}
\else
\documentclass[UTF8]{\@DocTypeCTEX}
\fi
\makeatother

% LiterateHaskell 设置
\makeatletter
\ifx \@NoLiterateHaskell \undefined
\usepackage{listings}
\usepackage{color,xcolor}
\newcounter{codeline}
\setcounter{codeline}{1}
\setcounter{lstnumber}{1}

\def\lstcodebgcolor{\color[rgb]{1.00,0.90,1.00}}
\def\lstcodekw{\bfseries\color[rgb]{0.16,0.53,0.53}}
\def\lstcodecm{\rmfamily\itshape}
\def\lstcodens{\tiny}
\def\lstspecbgcolor{\color[rgb]{0.96,0.97,0.38}}
\def\lstspeckw{\bfseries\color[rgb]{0.16,0.53,0.53}}
\def\lstspeccm{\rmfamily\itshape}
\def\lstspecns{\tiny}
% code 环境
\lstnewenvironment{code}[1][]
{ \lstset
  { firstnumber=last
  , language=Haskell
  , breaklines
  , backgroundcolor=\lstcodebgcolor
  , basicstyle=\sffamily
  , keywordstyle=\lstcodekw
  , commentstyle=\lstcodecm
  , flexiblecolumns
  , numbers=left
  , numberstyle=\lstcodens
  , frame=trBL
  , label=sourceCtr
  , #1
  }
}
{
  \setcounter{codeline}{\value{lstnumber}}
}
% spec 环境
\lstnewenvironment{spec}[1][]
{ \lstset
  { firstnumber=last
  , language=Haskell
  , breaklines
  , backgroundcolor=\lstspcebgcolor
  , basicstyle=\sffamily
  , keywordstyle=\lstspcekw
  , commentstyle=\lstspcecm
  , flexiblecolumns
  , numbers=left
  , numberstyle=\lstspcens
  , frame=trBL
  , label=sourceCtr
  , #1
  }
}
{
  \setcounter{codeline}{\value{lstnumber}}
}
% 设置SQL
%\lstnewenvironment{sql}
%{ \lstset
%    { firstnumber=last
%        , language=SQL
%        , breaklines
%        , backgroundcolor=\lstspcebgcolor
%        , basicstyle=\sffamily
%        , keywordstyle=\lstspcekw
%        , commentstyle=\lstspcecm
%        , flexiblecolumns
%        , numbers=none
%        , frame=trBL
%        , label=sourceCtr
%    }
%}
%{
%    \setcounter{codeline}{\value{lstnumber}}
%}
% ignore
\long\def\ignore#1{\relax}

% 设置 SQL 环境
\lstnewenvironment{sql}[1][]{\lstset{firstnumber=last, language=SQL, breaklines, basicstyle=\sffamily, keywordstyle=\lstcodekw, commentstyle=\lstcodecm, flexiblecolumns,#1}}{}
\fi
\makeatother

% 连接的设置
\usepackage[colorlinks,linkcolor=blue,anchorcolor=blue,citecolor=red,bookmarksnumbered]{hyperref}

%修改Chapter的格式
\makeatletter
\ifdefined\@NoStyleSection
\relax
\else
% 设置 section 格式
\usepackage{titlesec,titletoc}
\titleformat{\section}[display]{\Huge\bfseries}{\,\thesection\,}{1em}{}
\fi
\makeatother


% 设置附录的格式
\makeatletter
\ifx \@UsingAppendix \undefined
\usepackage[titletoc]{appendix}
\renewcommand\appendixname{附录}
\renewcommand\appendixname{附录}
\appendixtitleon
\fi
\makeatother

% 对版本的高亮
\def\apiver#1{\colorbox[rgb]{0.92,0.72,0.42}{#1}}

% 脚注随页码更新
\makeatletter
\ifx \@FnWithPage \undefined
\relax
\else
\usepackage{chngcntr}
\counterwithin{footnote}{page}
\fi

% caption
\usepackage{caption}

%%%%%%%%%%%%%%%%%%%%%%%%%%%%%%%%%%%%%%%%%%%%%%%%

