




% document/DindoDevNote.tex

% DingoBackend -- Dindo 开发总结报告


\documentclass{dingo}

\title{Dindo 开发总结报告}

% 定义封面颜色
\definecolor{dindoblue}{rgb}{0.00,0.60,1.00}

% 定义字体颜色
\definecolor{dindowhite}{rgb}{0.93,0.93,0.93}

\titlebgcolor{dindoblue}
\titlefontcolor{dindowhite}
\version{0.0.0.1}
\coverpic{Dingo-B}
\city{西安, Xi`an}
\author{李约瀚\\qinka@live.com\\14130140331}

\begin{document}
  %% 封面
  \makecover
  %% 前言
  \section*{前言}
   Dindo 是 Dingo 项目的后端,使用微服务架构
    \footnote{事实上,直接使用微服务架构就像直接使用野生木耳一样,是个糟糕的决定。由于在深刻的了解 微服务架构 之前,
    认为微服务架构是一个可以直接使用的东西。但是在了解之后,发现微服务架构使用的前提是在单体架构已经达到对业务逻辑划分
    完善的情况下使用。也就是为服务不是给帅气的小伙子使用的东西,而是老而成熟的男人是用的。微服务架构在一个创业时期(或者与之等同)
    的项目中是不应该是用的,或者说是,我们应该再有类似于数据库中的 几个范式与实体关系 相类似的 预备范式(pNF)、第一范式(1NF)与
    第二范式(2NF),和事件与数据驱动关系和调用关系。来指导为服务中最重要的业务模块与逻辑划分。}
  ,同时使用 Haskell 语言的 Yesod 框架作为后端的框架
    \footnote{这一部分的感悟会在之后详细讨论。}
  。后端开发计划使用 CI(Travis-CI),Coverage,Docker,CodeClimate
    \footnote{CodeClimate 一个代码质量分析软件,在线工具。}
  ,等工具。但是由于进度仅仅使用到了 Travis-CI 与 Docker。

  整个开发中,经可能遵循敏捷开发模型。然而有序开发时间是课余时间中的业余时间,因此 基本没有遵循开发模型,
  没有按时写周报什么的。

  然而 Dindo 是一个不错的创业点子,这一点必须感谢我们的铲平经理\footnote{这个称呼是团队中某个同志手抖打错而有的。}
  想到这样一个点子。当然必须承认,刘西洋老师才是这个点子的根本来源。而同时,这个项目,对我来时还是值得在这一个学期结束后
  继续做下去的。

  \subsection*{GitHub 上的状态}
    整个 DingoBackend ,不是整个 Dingo 的大部分内容都是开源的
      \footnote{剩下的没有开源的,只能说是大家太懒。唯一一个没有开源的是一个储存配置文件的仓库,为了安全,没有开源。
      可以访问 \href{https://github.com/DingoLab}{DingoLab}}
    。DingoBackend 至目前\footnote{这个文档编写期间},在 GitHub 上一共有 6 个分支:
    \begin{description}
      \item[master] 主要分支,也是我在某一段时间主要开发的分支。但在开发一段时间之后,发现这个分支并没有能满足开发要求的。
        于是开始发回版本控制软件的威力。当前的提交为:
        \href{https://github.com/DingoLab/DingoBackend/commit/1705b461b8c2d9aa2245c2725006771ad8065995}{1705b46}
      \item[documents-dev] 这个分支主要是负责文档\footnote{目前这个文档就在这个分支中}。当前提交为:
        \href{https://github.com/DingoLab/DingoBackend/commit/f5d76c0fc408337203fc1fd54393ef41372ac4ea}{f5d76c0}
      \item[dindo-dev] 这个分支是第一次重构时产生的分支,目前已停止开发。当前提交为:
        \href{https://github.com/DingoLab/DingoBackend/commit/235818a90dff38b8276ce53233c0cdaaff7acdf0}{235818a}
      \item[dindo-dev2] 这个分支是第二次重构时产生的分支,为第一次重构“失败”后再次重构产生的,为目前主要开发的分支。当前提交为:
        \href{https://github.com/DingoLab/DingoBackend/commit/3a6a37d1183b270685cee25ad031e707e5e3b7fd}{3a6a37d}
      \item[part2-development] 这个分支是重构之前,API 第二部分的开发分支,目前完成度一般。当前提交是\href{https://github.com/DingoLab/DingoBackend/commit/10fbe6e39a8cc58e68b247423043d9b9e9012471}{10fbe6e}
      \item[hoxfix-fork-yesod-persist-pg] 这个分支是在尝试解决引发重构的那个问题发生时,建立的 hotfix 分支,但由于原作者设计上的缺陷与个人经验的不足,暂时无法解决。于是重构。
      当前的提交为:\href{https://github.com/DingoLab/DingoBackend/commit/2255b7be491450b555b5ac1f58bcb932e909bd1a}{2255b7b}
    \end{description}
    \subsection*{DockerHub 上的镜像}
      Docker Hub 上的 Docker 镜像暂时只用于测试。仓库地址为 \href{https://hub.docker.com/r/qinka/dindo/}{qinka/dindo}。
      Docker 仓库中共大约有 100 个镜像版本,均由持续集成生成。大致有 GHC 编译器 8.0.1 与 7.10.3 两个版本
      镜像。
    \subsection*{实验所用服务器}
      实验与测试所用的服务器是5台位于软院科协的主机\footnote{主机的所有权为西安电子科技大学软件学院。
        SSSTA具有使用权限。其中主机被标号为0~5号。0号机是处于一种 Standby 状态。}
      搭建 Docker 集群中测试的。
      
  % 目录
  \newpage
  \makecontent
  
  % 开发中的总结
  \section{设计总结}
    整个 Dingo 的业务流程的设计不是我,然而还是得说目前的设计师糟糕透了。从业务流程的角度来看,无论是对于
    为服务架构,还是说对于一般的单体架构。这个划分不是十分的明确,业务逻辑比较混乱。同时传输数据的方式
    较为一般,传输的数据有时并不是十分的让人满意。
    
    在数据库设计方面,由于数据库课程的学习相对落后于业务流程的设计较为一般,整个数据库的设计,没有能证明
    满足一范式。同时部分地方由于业务设计考虑不充分,与其他外部因素一起导致了重构。
    
    在设计方面的感悟,十分简单。为服务架构的设计,大致应该分为两步:单体架构的设计与微服务化。
    
    首先设计方面,一开始不能直接按照微服务架构的“设计思路”来设计,这样会照成对某些“似人非人,似鬼非鬼”
    的内容经行频繁的划分调整,而且这种调整一般通常是在设计结束后的开发过程中的。而这种调整,会造成对代码的
    频繁的大幅度改动,而影响开发过程。同时一些不满足微服务架构思想的设计会使得设计的结果几乎与单体架构
    没有区别的状态。在单体架构的设计中,应该尝试将将业务逻辑划分清楚,但是不需要尝试划分。当单体架构中的
    业务逻辑充分清晰时候,就差不多可以分解为微服务架构。但是此时单一应用开发应该基本完毕,也就是说,
    单体应用在某种程度上来说是算做微服务架构的一种原型。
    
    微服务化的过程中由于之前单体架构的相对成熟,服务划分变的相对简单。然而微服务带来的一些“不好的”地方。
    例如,当服务拆分后每个服务的实例工程的依赖会变得不同,有时会出现一些依赖冲突。个人认为的解决方案
    首先是,使用一些类似于 Haskell 中的 Stackage 
    \footnote{Stackage 是由 Stable hackage 组成。Hackage is the Haskell community's central
        package archive of open source software。} 
    的相对稳定的依赖源。同时对于一些常用的包,可以使用某些特殊的方式进行包装,来提供一个稳定的
    “接口”。同时需要在开发的网络环境中架设一个仓库,以保证每一次的测试与交付的构建。
    
    在采用微服务架构的同时,应该使用一些与敏捷开发与微服务架构相匹配的工具来辅助开发。例如可以在
    开发工程中使用 \href{http://www.travis-ci.org}{Travis-CI} 这样的工具作为持续集成的工具。
    再比如,使用 CaaS \footnote{Container as a Server} 与 容器技术 \footnote{比如 Docker}
    作为从测试到交付、部署的媒介,以提供开发效率。同时,使用像 Code Climate 这样的工具做一些代码
    的检查,使用 VersionEye 对依赖经行评估,使用 Git 这样的版本控制系统来加速开发。多使用开源工具。
    同时贡献开源工具。
    
    微服务架构与 Docker 结合带来的是从开发到测试、交付、部署等方面的全面的自动化与高效率。
    
    \section{数据库设计总结}
    数据的设计方面来说,对于一个正常的应用,无论是否使用微服务架构,一个正常的思路是:数据库绝不可能是
    单一的数据库。一般来说是 集群级别的 SQL 数据库与 NoSQL 数据库作为几种缓存的大体设计。
    
    Dingo 的后端在一开始的时候,还是采用用单一的 PostgreSQL数据库作为数据存放的地方。虽然有扩展能力
    但是后期对 最一开始使用的 数据库驱动的测评的结果表明,这种扩展基本是十分难受的。在后两次的重构中
    考虑到了这一点,开始进行变化。然而唯一的不足时由于时间有限,并没有办法经行这一方面的改动。
    
    而同时由于数据库的课程进度,并没有按照第几范式的设计要求设计。甚至引发了重构。
    数据的在 Dindo 中的作用,或者说是 Dindo 的作用只是将 数据库有限并以相对安全的方式暴露给了客户端。
    
    \section{后端开发总结}
    开发过程相对容易,在业务划分好之后,按照约定的API 文档,开始编写代码。这个后端的代码(Haskell)部分
    采用 Literate Haskell 格式,将 \LaTeX 代码与 Haskell 结合在一起。虽然由于时间缘故,没能写太多注释
    但是还是有一定的注释效果的。
    
    同时本次开发中,使用到了 Stack 与 Stackage 来解决 Cabal 与 Hackage 中可能的版本冲突。测试应用到了
    Travis-CI 并且生成 Docker 文件。后期虽然由于 llvm 的 apt 仓库崩溃,无法使用  应用 llvm-3.6
    为编译后端的的GHC-8.0.1生成Docker 镜像。但是其他版本的还是正确 的生成。当然之后使用的一个 CTAN 镜像
    也崩溃了,使得\LaTeX 文档也无法生成了。
    
    \section{重构总结}
    由于种种原因,进行了重构。个人感受最深的是不要使用过度底层的框架与轮子,同时不要太相信一些,尝试通吃
    的轮子或者中间件。
\end{document}
