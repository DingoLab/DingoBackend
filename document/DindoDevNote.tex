




% document/DindoDevNote.tex

% DingoBackend -- Dindo 开发总结报告


\documentclass{dindo}

\title{Dindo 开发总结报告}

% 定义封面颜色
\definecolor{dindoblue}{rgb}{0.00,0.60,1.00}

% 定义字体颜色
\definecolor{dindowhite}{rgb}{0.93,0.93,0.93}

\titlebgcolor{dindoblue}
\titlefontcolor{dindowhite}
\version{0.0.3.0}
\coverpic{Dingo-A}
\city{西安, Xi`an}
\author{李约瀚\\qinka@live.com\\14130140331}

\begin{document}
  %% 封面
  \makecover
  %% 前言
  \section*{前言}
   Dindo 是 Dingo 项目的后端,使用微服务架构
    \footnote{事实上,直接使用微服务架构就像直接使用野生木耳一样,是个糟糕的决定。由于在深刻的了解 微服务架构 之前,
    认为微服务架构是一个可以直接使用的东西。但是在了解之后,发现微服务架构使用的前提是在单体架构已经达到对业务逻辑划分
    完善的情况下使用。也就是为服务不是给帅气的小伙子使用的东西,而是老而成熟的男人是用的。微服务架构在一个创业时期(或者与之等同)
    的项目中是不应该是用的,或者说是,我们应该再有类似于数据库中的 几个范式与实体关系 相类似的 预备范式(pNF)、第一范式(1NF)与
    第二范式(2NF),和事件与数据驱动关系和调用关系。来指导为服务中最重要的业务模块与逻辑划分。}
  ,同时使用 Haskell 语言的 Yesod 框架作为后端的框架
    \footnote{这一部分的感悟会在之后详细讨论。}
  。后端开发计划使用 CI(Travis-CI),Coverage,Docker,CodeClimate
    \footnote{CodeClimate 一个代码质量分析软件,在线工具。}
  ,等工具。但是由于进度仅仅使用到了 Travis-CI 与 Docker。

  整个开发中,经可能遵循敏捷开发模型。然而有序开发时间是课余时间中的业余时间,因此 基本没有遵循开发模型,
  没有按时写周报什么的。

  然而 Dindo 是一个不错的创业点子,这一点必须感谢我们的铲平经理\footnote{这个称呼是团队中某个同志手抖打错而有的。}
  想到这样一个点子。当然必须承认,刘西洋老师才是这个点子的根本来源。而同时,这个项目,对我来时还是值得在这一个学期结束后
  继续做下去的。

  \subsection*{GitHub 上的状态}
    整个 DingoBackend ,不是整个 Dingo 的大部分内容都是开源的
      \footnote{剩下的没有开源的,只能说是大家太懒。唯一一个没有开源的是一个储存配置文件的仓库,为了安全,没有开源。
      可以访问 \href{https://github.com/DingoLab}{DingoLab}}
    。DingoBackend 至目前\footnote{这个文档编写期间},在 GitHub 上一共有 6 个分支:
    \begin{description}
      \item[master] 主要分支,也是我在某一段时间主要开发的分支。但在开发一段时间之后,发现这个分支并没有能满足开发要求的。
        于是开始发回版本控制软件的威力。当前的提交为:
        \href{https://github.com/DingoLab/DingoBackend/commit/1705b461b8c2d9aa2245c2725006771ad8065995}{1705b46}
        。
      \item[documents-dev] 这个分支主要是负责文档\footnote{目前这个文档就在这个分支中}。当前提交为:
        \href{https://github.com/DingoLab/DingoBackend/commit/f5d76c0fc408337203fc1fd54393ef41372ac4ea}{f5d76c0}
      \item[dindo-dev] 这个分支是第一次重构时产生的分支,目前已停止开发。当前提交为:
        \href{https://github.com/DingoLab/DingoBackend/commit/235818a90dff38b8276ce53233c0cdaaff7acdf0}{235818a}
      \item[dindo-dev2] 这个分支是第二次重构时产生的分支,为第一次重构“失败”后再次重构产生的,为目前主要开发的分支。当前提交为:
        \href{https://github.com/DingoLab/DingoBackend/commit/3a6a37d1183b270685cee25ad031e707e5e3b7fd}{3a6a37d}
      \item[]
    \end{description}
\end{document}
