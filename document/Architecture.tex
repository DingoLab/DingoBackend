\documentclass[UTF8]{book}
\def\ARCHITECTURETEX{\relax}





% Preamble.tex

%%%
%%% 导言区
%%%

% 防止重复引用
\ifdefined \preambleInc
 % 如果重复引用
\setcounter{lstnumber}{1}
\endinput
\else
\def\preambleInc{}
\fi

%%%%%%%%%%%%%%%%%%%%%%%%%%%%%%%%%%%%%%%%%%%%%%%%

% Document Class 设置
\makeatletter
\ifx \@UseCTEX \undefined
\documentclass[UTF8]{\@DocType}
\usepackage{xeCJK}
\setCJKmainfont{WenQuanYi Zen Hei}
\else
\documentclass[UTF8]{\@DocTypeCTEX}
\fi
\makeatother

% LiterateHaskell 设置
\makeatletter
\ifx \@NoLiterateHaskell \undefined
\usepackage{listings}
\usepackage{color,xcolor}
\newcounter{codeline}
\setcounter{codeline}{1}
\setcounter{lstnumber}{1}

\def\lstcodebgcolor{\color[rgb]{1.00,0.90,1.00}}
\def\lstcodekw{\bfseries\color[rgb]{0.16,0.53,0.53}}
\def\lstcodecm{\rmfamily\itshape}
\def\lstcodens{\tiny}
\def\lstspecbgcolor{\color[rgb]{0.96,0.97,0.38}}
\def\lstspeckw{\bfseries\color[rgb]{0.16,0.53,0.53}}
\def\lstspeccm{\rmfamily\itshape}
\def\lstspecns{\tiny}
% code 环境
\lstnewenvironment{code}[1][]
{ \lstset
  { firstnumber=last
  , language=Haskell
  , breaklines
  , backgroundcolor=\lstcodebgcolor
  , basicstyle=\sffamily
  , keywordstyle=\lstcodekw
  , commentstyle=\lstcodecm
  , flexiblecolumns
  , numbers=left
  , numberstyle=\lstcodens
  , frame=trBL
  , label=sourceCtr
  , #1
  }
}
{
  \setcounter{codeline}{\value{lstnumber}}
}
% spec 环境
\lstnewenvironment{spec}[1][]
{ \lstset
  { firstnumber=last
  , language=Haskell
  , breaklines
  , backgroundcolor=\lstspcebgcolor
  , basicstyle=\sffamily
  , keywordstyle=\lstspcekw
  , commentstyle=\lstspcecm
  , flexiblecolumns
  , numbers=left
  , numberstyle=\lstspcens
  , frame=trBL
  , label=sourceCtr
  , #1
  }
}
{
  \setcounter{codeline}{\value{lstnumber}}
}
% 设置SQL
%\lstnewenvironment{sql}
%{ \lstset
%    { firstnumber=last
%        , language=SQL
%        , breaklines
%        , backgroundcolor=\lstspcebgcolor
%        , basicstyle=\sffamily
%        , keywordstyle=\lstspcekw
%        , commentstyle=\lstspcecm
%        , flexiblecolumns
%        , numbers=none
%        , frame=trBL
%        , label=sourceCtr
%    }
%}
%{
%    \setcounter{codeline}{\value{lstnumber}}
%}
% ignore
\long\def\ignore#1{\relax}

% 设置 SQL 环境
\lstnewenvironment{sql}[1][]{\lstset{firstnumber=last, language=SQL, breaklines, basicstyle=\sffamily, keywordstyle=\lstcodekw, commentstyle=\lstcodecm, flexiblecolumns,#1}}{}
\fi
\makeatother

% 连接的设置
\usepackage[colorlinks,linkcolor=blue,anchorcolor=blue,citecolor=red,bookmarksnumbered]{hyperref}

%修改Chapter的格式
\makeatletter
\ifdefined\@NoStyleSection
\relax
\else
% 设置 section 格式
\usepackage{titlesec,titletoc}
\titleformat{\section}[display]{\Huge\bfseries}{\,\thesection\,}{1em}{}
\fi
\makeatother


% 设置附录的格式
\makeatletter
\ifx \@UsingAppendix \undefined
\usepackage[titletoc]{appendix}
\renewcommand\appendixname{附录}
\renewcommand\appendixname{附录}
\appendixtitleon
\fi
\makeatother

% 对版本的高亮
\def\apiver#1{\colorbox[rgb]{0.92,0.72,0.42}{#1}}

% 脚注随页码更新
\makeatletter
\ifx \@FnWithPage \undefined
\relax
\else
\usepackage{chngcntr}
\counterwithin{footnote}{page}
\fi

% caption
\usepackage{caption}

%%%%%%%%%%%%%%%%%%%%%%%%%%%%%%%%%%%%%%%%%%%%%%%%



%后端宏观架构
\part{后端宏观架构}

此部分将包含处理负载均衡的大致方法、弹性计算的解决方案、后端API与服务程序分割的
策略,其中会有另一个文档详细描述如何测试 Yesod 框架的负载特性。
此外还将会有对整个宏观设计一个小的对可行性的讨论。


\chapter{负载均衡}
\section{预期目标}
Dingo 的后端的使用人数并不会多,而且这就是个大作业。然而 按着刘西洋教授的号召:代入感,我还是计划提前预留好
应对大量负载的东西。而且放一颗卫星,希望所能完成的负载量能超过一千万。但是从实际来讲,可能一千万的负载量在
绝大多数的情况下是不需要的,甚至有些时候的负载会长时间达到个位数级别。所以,整个负载是基于弹性计算的。
同时在阅读到 灵雀云微信公众号推的一篇文章,有关于微服务的。大致有了这样整个负载的解决方案。
\section{硬件负载部分}
我需要看门见山的点出,整个负载均衡的处理办法不包括硬件级别的负载均衡,而之所以在这样一部分点出来,是因为要点出为何不使用硬件均衡。

不使用硬件均衡的原因很简单,那就是由于会使用到弹性计算,所以可能、应该用不上,也没有办法去用。
整个负载计划是使用负载均衡软件与、服务注册与Docker。而整个后端是直接使用 容器,而非直接使用物理或虚拟机或传统意义上的云计算。
\section{负载的组成}
负载的组成有四大部分组成:负载均衡、服务注册、微服务、DockerShip。

负载均衡使用的是多层 Nginx 进行负载均衡。使用 Docker 容器中运行的 Nginx 。而假设对一个标准的容器,我们定义一个标准单位。
定义标准 CPU 单位计算能力$c$,并使得该单位为离散的(即存在两个数值,其之间不存在任何合法数值)。设定函数
$f_{Nginx}(c,m)$ 表示对一个 Docker容器“运行” Nginx 镜像,其中的容器所分配的 cpu 资源为 c,内存配大小为 m,
并假定其他参数不影响。这个 Docker 容器,$f_{Nginx}$所表示的则是该情况之下 Nginx 的最大负载量。并设 $d_{Nginx}(c,m)$ 表示该情况下的延迟。

