




% Architecture.tex

%%%% Architecture

%%%%%%%%%%%%%%
%%% 导言区
%%%%%%%%%%%%%%

% flag
\makeatletter
\def\@NoStyleChaper{\ralex} % 设置不适用章节
\def\@ARCHDoc{\ralex} % 设置文档标签
\def\@UsingAppendix{\ralex} % 使用附录
\def\@DocType{article}
\def\@DocTypeCTEX{ctexart}
\def\primarykey#1{\colorbox[rgb]{0.9,0.7,0.65}{#1}}
\makeatother

% 导入导言区





% Preamble.tex

%%%
%%% 导言区
%%%

% 防止重复引用
\ifdefined \preambleInc
 % 如果重复引用
\setcounter{lstnumber}{1}
\endinput
\else
\def\preambleInc{}
\fi

%%%%%%%%%%%%%%%%%%%%%%%%%%%%%%%%%%%%%%%%%%%%%%%%

% Document Class 设置
\makeatletter
\ifx \@UseCTEX \undefined
\documentclass[UTF8]{\@DocType}
\usepackage{xeCJK}
\setCJKmainfont{WenQuanYi Zen Hei}
\else
\documentclass[UTF8]{\@DocTypeCTEX}
\fi
\makeatother

% LiterateHaskell 设置
\makeatletter
\ifx \@NoLiterateHaskell \undefined
\usepackage{listings}
\usepackage{color,xcolor}
\newcounter{codeline}
\setcounter{codeline}{1}
\setcounter{lstnumber}{1}

\def\lstcodebgcolor{\color[rgb]{1.00,0.90,1.00}}
\def\lstcodekw{\bfseries\color[rgb]{0.16,0.53,0.53}}
\def\lstcodecm{\rmfamily\itshape}
\def\lstcodens{\tiny}
\def\lstspecbgcolor{\color[rgb]{0.96,0.97,0.38}}
\def\lstspeckw{\bfseries\color[rgb]{0.16,0.53,0.53}}
\def\lstspeccm{\rmfamily\itshape}
\def\lstspecns{\tiny}
% code 环境
\lstnewenvironment{code}[1][]
{ \lstset
  { firstnumber=last
  , language=Haskell
  , breaklines
  , backgroundcolor=\lstcodebgcolor
  , basicstyle=\sffamily
  , keywordstyle=\lstcodekw
  , commentstyle=\lstcodecm
  , flexiblecolumns
  , numbers=left
  , numberstyle=\lstcodens
  , frame=trBL
  , label=sourceCtr
  , #1
  }
}
{
  \setcounter{codeline}{\value{lstnumber}}
}
% spec 环境
\lstnewenvironment{spec}[1][]
{ \lstset
  { firstnumber=last
  , language=Haskell
  , breaklines
  , backgroundcolor=\lstspcebgcolor
  , basicstyle=\sffamily
  , keywordstyle=\lstspcekw
  , commentstyle=\lstspcecm
  , flexiblecolumns
  , numbers=left
  , numberstyle=\lstspcens
  , frame=trBL
  , label=sourceCtr
  , #1
  }
}
{
  \setcounter{codeline}{\value{lstnumber}}
}
% 设置SQL
%\lstnewenvironment{sql}
%{ \lstset
%    { firstnumber=last
%        , language=SQL
%        , breaklines
%        , backgroundcolor=\lstspcebgcolor
%        , basicstyle=\sffamily
%        , keywordstyle=\lstspcekw
%        , commentstyle=\lstspcecm
%        , flexiblecolumns
%        , numbers=none
%        , frame=trBL
%        , label=sourceCtr
%    }
%}
%{
%    \setcounter{codeline}{\value{lstnumber}}
%}
% ignore
\long\def\ignore#1{\relax}

% 设置 SQL 环境
\lstnewenvironment{sql}[1][]{\lstset{firstnumber=last, language=SQL, breaklines, basicstyle=\sffamily, keywordstyle=\lstcodekw, commentstyle=\lstcodecm, flexiblecolumns,#1}}{}
\fi
\makeatother

% 连接的设置
\usepackage[colorlinks,linkcolor=blue,anchorcolor=blue,citecolor=red,bookmarksnumbered]{hyperref}

%修改Chapter的格式
\makeatletter
\ifdefined\@NoStyleSection
\relax
\else
% 设置 section 格式
\usepackage{titlesec,titletoc}
\titleformat{\section}[display]{\Huge\bfseries}{\,\thesection\,}{1em}{}
\fi
\makeatother


% 设置附录的格式
\makeatletter
\ifx \@UsingAppendix \undefined
\usepackage[titletoc]{appendix}
\renewcommand\appendixname{附录}
\renewcommand\appendixname{附录}
\appendixtitleon
\fi
\makeatother

% 对版本的高亮
\def\apiver#1{\colorbox[rgb]{0.92,0.72,0.42}{#1}}

% 脚注随页码更新
\makeatletter
\ifx \@FnWithPage \undefined
\relax
\else
\usepackage{chngcntr}
\counterwithin{footnote}{page}
\fi

% caption
\usepackage{caption}

%%%%%%%%%%%%%%%%%%%%%%%%%%%%%%%%%%%%%%%%%%%%%%%%

 % Preamble.tex

\title{后端架构}
\author{李约瀚 \\ qinka@live.com \\ DingoLab \\ 14130140331}

\begin{document}
\maketitle
\newpage

%后端架构
\section{后端架构}

此部分将包含处理负载均衡的大致方法、弹性计算的解决方案、后端API与服务程序分割的
策略,其中会有另一个文档详细描述如何测试 Yesod 框架的负载特性。
此外还将会有对整个宏观设计一个小的对可行性的讨论。


\subsection{负载均衡}
\subsubsection{预期目标}
Dingo 的后端的使用人数并不会多,而且这就是个大作业。然而 按着刘西洋教授的号召:代入感,我还是计划提前预留好
应对大量负载的东西。而且放一颗卫星,希望所能完成的负载量能超过一千万。但是从实际来讲,可能一千万的负载量在
绝大多数的情况下是不需要的,甚至有些时候的负载会长时间达到个位数级别。所以,整个负载是基于弹性计算的。
同时在阅读到 灵雀云微信公众号推的一篇文章,有关于微服务的。大致有了这样整个负载的解决方案。
\subsubsection{硬件负载部分}
我需要看门见山的点出,整个负载均衡的处理办法不包括硬件级别的负载均衡,而之所以在这样一部分点出来,是因为要点出为何不使用硬件均衡。

不使用硬件均衡的原因很简单,那就是由于会使用到弹性计算,所以可能、应该用不上,也没有办法去用。
整个负载计划是使用负载均衡软件与、服务注册与Docker。而整个后端是直接使用 容器,而非直接使用物理或虚拟机或传统意义上的云计算。
\subsubsection{负载的组成}
负载的组成有四大部分组成:负载均衡、服务注册、微服务、DockerShip。

负载均衡使用的是多层 Nginx 进行负载均衡。使用 Docker 容器中运行的 Nginx 。而假设对一个标准的容器,我们定义一个标准单位。
定义标准 CPU 单位计算能力$c$,并使得该单位为离散的(即存在两个数值,其之间不存在任何合法数值)。设定函数
$f_{nginx}(c,m)$ 表示对一个 Docker容器“运行” Nginx 镜像,其中的容器所分配的 cpu 资源为 c,内存配大小为 m,
并假定其他参数不影响。这个 Docker 容器,$f_{nginx}$所表示的则是该情况之下 Nginx 的最大负载量。并设 $d_{nginx}(c,m)$ 表示该情况下的延迟。
假设当前的负载量为$a$ 则$\lceil \log_{f_{nginx}(c,m)}a\rceil$ 为此时所需的 Nginx 层数,$\lceil \log_{f_{nginx}(c,m)}a\rceil \times d_{nginx}(c,m)$ 为此时的延迟。对于 单独\footnote{一般来说,Docker 容器中只会运行一种服务} 
运行在 CPU 计算能力为 $c$,内存为$m$的容器中的服务\footnote{取所有服务的平均},设其最大负载\footnote{设此处最大负载为相对于单负载延迟,延迟降为其延迟的$\frac{6}{5}$时候对应的负载量}为
 $f_{yesod}(c,m)$,并设此时的延迟为$\frac{6}{5} \times d_{yesod}(c,m)$。则最后负载之后整个服务的延迟的平均最大值是$$\lceil \log_{f_{nginx}(c_1,m_1)}a\rceil \times d_{nginx}(c_1,m_1)+\frac{6}{5} \times d_{yesod}(c_2,m_2)$$

服务注册目前还没有具体选定使用那个,但是还是给出了一些限定。
首先服务注册将需要使用 Redis 这样的内存中的数据库,来用做缓存。
同时会需要有注册、Heartbeat、与销毁的 API 以供容器注册、持续与销毁。
同时,能与 Nginx 或者被修改过代码的Nginx 共同工作。同时整个服务注册对于Nginx 的查询
的延时对整个请求的时间影响不会大于 $d_{nginx}(c,m)$。

微服务使得后端的服务能划分成适度的粒度,不仅能解决服务过于庞大复杂所带来的的问题,
同时,关键的一个问题是:对微服务架构对于弹性计算与均衡负载十分友好。
对于划分粒度适当的微服务架构的服务,当对于某一个或某一组的 API 的访问数量较大时,
起可以相对比较容易的增加增加处理这个或该组API 的那个服务模块,对于Docker 我们直接以几乎微小的
代价启动一个新的容器,而不必启动一个庞大的服务进程之类的东西,也没有虚拟机拖沓的启动。
同时不必为少数的API大量访问而增加整个服务。

Docker 还是一个比较新的而十分强的的工具,Docker 是一种虚拟化产品,而非虚拟机。
这就意味着虽然Docker 仅能运行 Linux 程序制作的镜像,然而其效率、大小与性能等方面与物理机的
差别十分小。而不同物理机上的大量的Docker容器可以组成一个大的虚拟网络,方便负载均衡与弹性计算。最关键的是能精确降低成本。

Docker 与微服务在整个服务的热更新中有这很好的友好度。通过服务注册,自动将过期的Docker容器下线,脱离平台,同时在有新的版本的镜像上线到镜像仓库是,一旦需要新增服务实例则能依赖Docker 的机制上线最新的镜像。当允许不同版本的镜像同时在线时,这种热替换能使得与后端服务相紧密贴合的一些部分都不会察觉到版本的更迭。

而对于后端服务于API 的分割主要需要考虑如下内容:
\begin{itemize}
    \item API 访问时的相关度
    \item 热更新是的粘黏度
    \item 业务与逻辑的扩展性
\end{itemize}
如果说传统的服务组织结构是一种固体的结构,那么微服务架构则可以算作一个液体的结构。
可以方便的改变形状,改变多少。API 的访问相关度就定了某几个 API 之间的关联程度,使得相关的API
处于同一容器之中,避免粒度过小。而不相关的API 则可以分开放置,保证一种相对独立干净的状态。
而考虑热更新的粘黏度则是因为在某一API跟新之后,或许会带来其他API 的变化,而导致为应对
此变化而进行更新,最后使得更新变成类似涟漪的状态。而业务与逻辑的扩展则是应对服务于业务的变动带来的不可预知的对不同API 之间的一些影响。

\subsection{弹性计算}
整个弹性计算的部分,使用 Docker 容器 与限定容器生存周期的策略。
当不考虑“允许响应时间延长的情况下”
假设当前有$n$ 个容器\footnote{再次仅考虑 Docker 容器,对于均衡负载部分与服务发现部分的影响暂时不考虑。},每个容器负载能力为$f$。
则当前理论的负载能力是$n \times f$。若设设当前实际的负载量为 $l$,则对于应对负载的需求有如下策略:
\subsubsection{无容忍方式的弹性调节}
最基本的做法是,使得实际负载的负载量至于 $((n-1) \times f,n \times f)$ 的区间之中,当实际负载大于这个区间的最大值时,增加一个容器,
当小于这个区间的最小值是,则减少一个容器。

这种调节方式较为机器,无法应对负载的突然剧增与突然减少。同时由于容器的启动尽管是分快速,然而并不是绝对为0,则当$\delta l$大于$f$时,
就会使得部分容器负载超量,造成与最起初的 假设“不允许响应时间延长的情况下” 的相违背。同时当只是一瞬间的负载,新增的容器则将在未被使用
的情况之下被删除,由于一瞬间的负载。
\subsubsection{设置缓冲区间的弹性调节}
仿照 C++ 的 STL 模板库中 vector<> 模板对于元素调节的方式,我们设置缓冲区。我们假设所允许的 $l$相对浮动的范围为 $(p_l(n,f),p_u(n,f))$
\footnote{计算出来应该是百分比}$p_l(n,f)$与$p_u(n,f)$为两个与当前容器的个数和负载能力相关的函数,分别反映的是当前允许的实际负载量与
当前最大负载量 $n \times f$ 的比值的最小值与最大值。当其比值超出这个范围,负责弹性调节的部分则通过事先计算好的方式进行调节。
比如当超过此上限时,容器数量增加 $10\%$。

这样的一种弹性调节方式,当配合数理统计的时候,可以解决第一种方法中的一些问题。当负载量有限的猛增时与有限的突然减少的时候,
对于整个系统的形象相对来说是比较小的。但还是可能出现负载怎家的或减少的量超出预期的情况与部分容器在未被使用的时候就被销毁。
然而相对第一种方式,设置缓冲区的 方式能在最好的利用弹性计算与实际负载效果最佳之间做出一个比较好的选择。

然而这个方案还存在一个问题就是对于函数 $p_l(n,f)$与$p_u(n,f)$ 的确定需要建立一定的数学模型,同时做一定量的实验,才能确定。
如果直接确定定值的话,或许是个糟糕的方法。

当然,令$p_l(n,f)=\frac{n-1}{n}$ 与 $p_u(n,f)=1$ 时,这个办法就退化成了第一种方法。
\subsubsection{通过机器学习的弹性调整}
使用机器学习调整的依据十分简单与现实。以 Dingo 为例子,Dingo 这种代收快的的 O2O 应用程序的访问高峰绝对不会出现在晚上,因为那个时间段
快递公司都不一定会派送快递的。Dingo 的高峰很可能会出现在学校快下课的时候或者其他时候。当然我现在对这个问题了解的深刻,我们还是换一个例子。
比如做O2O外卖的应用,外卖下单的高峰期就是在吃饭前一两个小时\footnote{这里的估算是根据平时个人的经验与身边其他人的习惯,再次本人并没有见过具体数据。}
换句话说,在只进行数理统计的情况下,当定外卖的人数这样一个样本庞大、其他因素基本不影响这个统计时,每天订外卖的人数(负载)与时间的统计曲线
基本是一致的,准确的说应该是基本相似的。然而譬如天气,物价等因素也会影响这样的一个曲线分布,而且得出这样一个曲线需要大量客观的数理统计。
所以在此提出使用机器学习的方式获取这样的一个曲线,来指导整个弹性调节中容器实例的需求。通过一个适当的学习算法,将比较大量的实际影响“负载”的
因素 “算计” 进去。使得预测的负载曲线与实际的曲线能比较好的吻合,并且有效的应对负载瞬间的增多与减少。

这种方法能避免之前两个方案所带来的问题,当然也会有少数时候,无法应对。同时,将机器学习的那个曲线固定为第二种方法中的两个函数,
则此方法就会再次退化为第二种方法。

\subsubsection{再次推广弹性调节}
上述三种方法均是建立在“不允许响应时间减慢,不考虑负载平衡、服务发现,同时也不考虑数据库性能” 的条件下产生的。
如果将上述因素考虑进去,则面对的是一个多维的优化问题或者类似的数学问题,需要建立一些数学模型,而由于软工大作业的重心在于软件工程的这方面,
弹性计算的这方面暂且考虑这么多。而同时基于,或者应该用“寄”这个字,弹性计算这个方案还解决了热更新等问题。

\subsection{API 与 服务的分割}
API 与 服务的分割 类似于 并行计算中粒度的划分。API 划分的过于细小,容器、负载均衡、服务发现等方面的某些资源会发生浪费,但是划分过大,
则会使得整个为服务的效能下降。

在此需要的是一些数学模型、与实际测试。

\section{API 及与客户端交互的方式组织的设计}
这部分是设计 API 与 和客户端交互方式的部分。
\section{数据库设计}
这部分的基本与软件工程的课程无关,然而考虑到本学期另一们课程——数据库系统概论。
还是需要详细讨论。

尽力不给苏向阳老师丢人,所以数据库还是需要好好设计的。这部分将包含 对所需数据库的种类与具体使用的 DBMS\footnote{Database Management System,这个项目中使用 PostgreSQL。}。
同时还有对数据库中存放的数据的具体内容,形式\footnote{表的组织方式。}等方面的分析与讨论。最后会尝试讨论为应对后端的那种液态架构,数据库的
架构\footnote{这里指的是集群等。}应该是怎样的。
\subsection{用户管理部分的数据库设计}
这部分主要是关于数据库部分中用户管理部分的内容,包括账户的基本信息,用户的基本信息。
账户基本信息,包括一个账户的唯一的id,密码。
\begin{sql}[caption=创建账户的表 table\_account]
    CREATE TABLE table_account
        ( key_uid CHAR(64) NOT NULL PRIMARY KEY
        , key_pash CHAR(64) NOT NULL CHECK( char_length(key_pash) = 64)
        )
    ;
\end{sql}
$key\_uid$ 是用户的姓名,$key_pash$ 为用户的密码的散列值\footnote{使用SHA256散列算法。}

\begin{sql}[caption=创建用户表 table\_usr]
    CREATE TABLE table_usr
        ( key_uid CHAR(64) NOT NULL PRIMARY KEY REFERENCES table_account(key_uid)
        , key_tel INT NOT NULL UNIQUE CHECK( key_tel > 10000000000 AND key_tel < 20000000000)
        , key_name CHAR(64) NOT NULL UNIQUE
        , key_email TEXT NOT NULL
        , key_rname CHAR(64) NOT NULL
        , key_prcid CHAR(18) NOT NULL CHECK(char_length(key_prcid) =18)
        , key_addr TEXT NOT NULL
        , key_status CHAR(1) NOT NULL CHECK(ARRAY[ascii(key_status)] <@ ARRAY[80,78,78]
        )
    ;
\end{sql}
$key\_tel$ 是用户的手机,$key\_name$ 是用户的用户名,$key\_email$ 是用户的电子邮件,
$key\_rname$是用户的真是姓名,$key\_prcid$ 是用户的身份证号,$key\_addr$是用户的地址。

\begin{sql}[caption=创建账户照片表 table\_apic]
    CREATE TABLE table_apic
        ( key_pic_id CHAR(64) NOT NULL PRIMARY KEY
        , key_uid CHAR(64) NOT NULL REFERENCES table_account(key_uid)
        , binary_pic BYTEA NOT NULL
        , key_type INT DEFAULT 0
        )
    ;
\end{sql}
其中 $key\_pic\_id$ 是图片的id,
$key\_uid$ 是一个外键,账户的id,
$binary\_pic$ 是图片的二进制数据,
$key\_type$ 是图片的类型\footnote{可为空,为空默认为0}。
\begin{sql}[caption=创建用户收货地址表 table\_addr]
	CREATE TABLE table_addr
		( key_aid CHAR(64) NOT NULL PRIMARY KEY
		, key_uid CHAR(64) NOT NULL REFERENCES table_account(key_uid)
		, key_p CHAR(64) NOT NULL 
		, key_c CHAR(64) NOT NULL
		, key_d CHAR(64) NOT NULL
		, key_b CHAR(64) NOT NULL
		, key_v CHAR(64) NOT NULL
		, key_zip INT NOT NULL CHECK(key_zip <1000000 AND key_zip > 99999)
		)
\end{sql}
\subsection{任务管理部分的数据库设计(代收快递)}
\begin{sql}[caption=创建任务关联表 table\_task]
    CREATE TABLE table_task
        ( key_tid CHAR(64) NOT NULL PRIMARY KEY
        , key_ca CHAR(64) NOT NULL REFERENCES table_account(key_uid)
        , key_cb CHAR(64) NOT NULL REFERENCES table_account(key_uid)
        )
    ;
\end{sql}

\begin{sql}[caption=创建任务信息表 table\_task\_info]
    CREATE TABLE table_task_info
        ( key_tid CHAR(64) NOT NULL PRIMARY KEY REFERENCES table_task(key_tid)
        , key_ew REAL NOT NULL
        , key_ns REAL NOT NULL
        , key_r REAL NOT NULL
        , key_w REAL NOT NULL
        , key_size REAL[] NOT NULL CHECK(array_length(key_size,1) = 3)
        , key_note TEXT
        , key_cost INT NOT NULL
        , key_des TEXT
        )
    ;
\end{sql}

\begin{sql}[caption=创建任务议价表 table\_task\_cost]
    CREATE TABLE table_task_cost
        ( key_tid CHAR(64) NOT NULL PRIMARY KEY REFERENCES table_task(key_tid)
        , key_ad INT[] NOT NULL DEFAULT ARRAY[]
        , key_bd INT[] NOT NULL DEFAULT ARRAY[]
        )
\end{sql}

\begin{sql}[caption=创建可代收表 table\_dd]
    CREATE TABLE table_dd
        ( key_did CHAR(64) NOT NULL PRIMARY KEY
        , key_dd TEXT NOT NULL
        , key_ew REAL NOT NULL
        , key_ns REAL NOT NULL
        , key_r REAL NOT NULL
        )
    ;
\end{sql}

\begin{sql}[caption=创建用户临时Token表 table\_tmptoken]
    CREATE TABLE table_tmptoken
        ( key_tmptoken CHAR(128) NOT NULL PRIMARY KEY
        , key_timeup TIMESTAMP WITH TIME ZONE NOT NULL
        , key_uid CHAR(64) NOT NULL REFERENCES table_account(key_uid)
        )
    ;
\end{sql}
\subsection{数据库视图与索引的设计}
\begin{sql}[caption=创建登陆视图 view_login]
	CREATE VIEW view_login (uid,uname,utel,pash) AS
		SELECT A.key_uid, U.key_name, U.key_tel, A.key_pash
		FROM table_account AS a, table_usr AS u
		WHERE A.key_tid = U.key_tid
	;
\end{sql}
\begin{sql}[caption=创建登陆用的索引]
	CREATE INDEX CONCURRENTLY index_login_tel ON table_usr(key_tel);
	CREATE INDEX CONCURRENTLY index_login_name ON table_usr(key_name);
\end{sql}
\begin{sql}[caption=创建]
\end{sql}
\end{document}
