




%  document/Dindo.tex

%% Dingo 后端 Dindo 的手册性质的文档

\documentclass{dingo}

\title{Dindo Document}

% 定义封面颜色
\definecolor{dindoblue}{rgb}{0.00,0.60,1.00}

% 定义字体颜色
\definecolor{dindowhite}{rgb}{0.93,0.93,0.93}

\titlebgcolor{dindoblue}
\titlefontcolor{dindowhite}

\coverpic{Dingo-A}
\city{西安, Xi`an}
\author{李约瀚\\qinka@live.com\\14130140331}
\begin{document}
    %封面
	\makecover
    %前言
    \section*{前言}
    这个文档是关于 Dingo 后端 Dindo 的说明性质的、手册性质的文档。这个文档包含对后端的一些说明,事例代码,与后端的实际主要的 Haskell
    代码。后端是采用 Haskell \footnote{\href{https://www.haskell.org}{Haskell} 是一门纯函数式的编程语言。},与 Yesod 
    \footnote{\href{http://www.yesodweb.com}{Yesod}Yesod 是一个使用 Haskell作为主要语言,的 RESTful API 的WEB 应用框架。}
    编写。使用的是Haskell 的文学编程方式,源代码为 \LaTeX 与 Haskell 混排的方式组织的\footnote{但由于Haskell 编译器与\LaTeX 系统的一些特性,
        文学编程的代码并不需要特殊工具,而直接可以被识别。因此较方便的将后端的代码编入文档。}
    
    Dingo 后端的名称被确定为 Dindo,当然这个行为是后端的负责者\footnote{也就是笔者。}自己定的。其来源是 笔者在数模校赛期间使用
    Lingo 是有个报错是关于 Lindo 的于是就开始称后端的程序为 Dindo。
    
    这个关于 Dindo 的文档,包括了Dingo 后端 Dindo 的架构设计、后端API代码与解释、Dindo运行指南等部分。
    Dindo 架构设计 中包含了 关于 Dindo 的设计的内容,与相应的一些解释。后端代码中包括了后端 每个API 的实现代码
    ,包括相关内容的代码,与API 调用等的说明。Dindo 运行指南 包括了,如何在“标准”的条件下,从零部署 Dindo。
    同时附录及其他章节还包括了一些其他的内容。
    %目录
    \newpage
	\makecontent
    \section{后端架构设计概论}
    \section{均衡负载设计}
    \section{弹性计算设计}
    \section{微服务架构设计}
    \section{数据库设计}
    \section{后端其他设计}
  \section{Dindo 公共组件}
  这部分是关于 Dindo 的公共组件的。由于 Dingo 后端采用的微服务架构\footnote{后面随时可能会称之为 微架构 。},不同的微服务之间,会有包括
  服务发现\footnote{目前的版本并没有开发实际的服务发现的内容,直接使用 Nginx 进行做均衡负载等。}、数据库
  \footnote{这一部分单独出来的。}、授权认证等是共用的。所以为了减少代码的重复使用,则独立出这一部分。
  
  \section{Dindo 数据库}
  
  \section{Dindo Launcher}
  
  \section{Dindo 微服务组件——用户管理}
  
  %附录
  \newpage
  \begin{appendix}
	  	% 备忘性质的
	  	\section{Docker 中 Weave  的配置} % 参考 http://debugo.com/docker-weave/
	  	Weave 是能将Docker 中每个物理主机中的连接起来一个工具,也就是能使的 Docker 容器跨主机互联。
	  	下面是配置(安装)Weave 的Shell 脚本:
	  	\begin{shell}[caption=Weave 安装]
#!/bin/sh
wget -O /usr/local/bin/weave \
https://github.com/zettio/weave/releases/download/latest_release/weave
chmod a+x /usr/local/bin/weave
dao pull weaveworks/weave:1.5.1
dao pull weaveworks/plugin:1.5.1
dao pull weaveworks/weaveexec:1.5.1
apt-get update
apt-get install bridge-utils
dao pull weaveworks/weavedb:latest
weave launch 192.168.1.181
	  	\end{shell}
	  	运行容器需要 使用
	  	\begin{shell}
$ weave run
	  	\end{shell}
  \end{appendix} 
  
\end{document}
