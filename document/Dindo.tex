




%  document/Dindo.tex

%% Dingo 后端 Dindo 的手册性质的文档

\documentclass{dingo}

\title{Dindo Document}

% 定义封面颜色
\definecolor{dindoblue}{rgb}{0.00,0.60,1.00}

% 定义字体颜色
\definecolor{dindowhite}{rgb}{0.93,0.93,0.93}

\titlebgcolor{dindoblue}
\titlefontcolor{dindowhite}

\coverpic{}
\city{西安, Xi`an}
\author{李约瀚\\qinka@live.com\\14130140331}
\begin{document}
	\makecover
	\makecontent
  \section{说明}
  这个文档是关于 Dingo 后端 Dindo 的说明性质的、手册性质的文档。这个文档包含对后端的一些说明,事例代码,与后端的实际主要的 Haskell
  代码。后端是采用 Haskell \footnote{\href{https://www.haskell.org}{Haskell} 是一门纯函数式的编程语言。},与 Yesod 
  \footnote{\href{http://www.yesodweb.com}{Yesod}Yesod 是一个使用 Haskell作为主要语言,的 RESTful API 的WEB 应用框架。}
  编写。使用的是Haskell 的文学编程方式,源代码为 \LaTeX 与 Haskell 混排的方式组织的\footnote{但由于Haskell 编译器与\LaTeX 系统的一些特性,
  	文学编程的代码并不需要特殊工具,而直接可以被识别。因此较方便的将后端的代码编入文档。}
  
  Dingo 后端的名称被确定为 Dindo,当然这个行为是后端的负责着\footnote{也就是笔者。}自己定的。其来源是 笔者在数模校赛期间使用
  Lingo 是有个报错是关于 Lindo 的于是就开始称后端的程序为 Dindo。
  
  \section{Dindo 公共组件}
  这部分是关于 Dindo 的公共组件的。由于 Dingo 后端采用的微服务架构\footnote{后面随时可能会称之为 微架构 。},不同的微服务之间,会有包括
  服务发现\footnote{目前的版本并没有开发实际的服务发现的内容,直接使用 Nginx 进行做均衡负载等。}、数据库
  \footnote{这一部分单独出来的。}、授权认证等是共用的。所以为了减少代码的重复使用,则独立出这一部分。
  
  \section{Dindo 数据库}
  
  \section{Dindo Launcher}
  
  \section{Dindo 微服务组件——用户管理}
\end{document}
